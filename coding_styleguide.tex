\documentclass{article}

\usepackage[ngerman]{babel}
\usepackage[utf8]{inputenc}
\usepackage{hyperref}

\begin{document}

\section{Coding Styleguide}

\subsection{Einleitung}

Die Python-Community legte schon von Beginn an viel Wert auf Lesbarkeit und
Konsistenz von Source Code. Dazu gehört auch ein einheitlicher Code-Stil.  Guido
van Rossum, der Autor von Python, schrieb deshalb seine Vorstellungen von
sauberem Code in einem \textit{Style Guide for Python Code} nieder. Dieser Style
Guide wurde im Jahr 2001 als Python Enhancement Proposal 8 -- kurz PEP8 --
veröffentlicht\footnote{\url{https://python.org/dev/peps/pep-0008/}}.

Der PEP8 Style Guide hat seit dann beinahe universelle Verbreitung gefunden.
Einer der zentralsten Punkte daraus -- die Verwendung von 4 Spaces anstelle von
Tabs -- wird gemäss einem
Analysetool\footnote{\url{http://sideeffect.kr/popularconvention\#python}} in
95\% der Projekte auf Github so umgesetzt. Im Rahmen dieser Bachelorarbeit
werden wir daher auch gemäss diesen Richtlinien arbeiten, mit einigen kleinen
Anpassungen.

\subsection{Coding Guidelines}

PEP8 ist in unserer Software verbindlich, mit folgenden Ausnahmen:

\begin{itemize}
	\item Maximale Zeilenlänge ist 99 Zeichen, nicht 79. Heutige Bildschirme sind
		viel grösser als frühere Terminals, es ergibt keinen Sinn Code umzubrechen
		um innerhalb der 80-Zeichen-Grenze zu bleiben wenn dadurch der Code weniger
		gut lesbar wird.
	\item Folgende Einrückungsregeln in den Code-Checking-Tools können in gewissen
		Fällen zu schlechter lesbarem Code führen und können deshalb ignoriert
		werden: \textit{E126}, \textit{E127}, \textit{E128}.
\end{itemize}

\subsection{Tools}

\subsubsection{Flake8}

Flake8 (\url{https://flake8.readthedocs.org/en/2.0/}) verbindet das Style
Checking Tool pep8\footnote{\url{https://pypi.python.org/pypi/pep8}} mit dem
Static Code Analysis Tool
pyflakes\footnote{\url{https://pypi.python.org/pypi/pyflakes}}. Für unser
Projekt kann folgende Konfiguration (\texttt{~/.config/flake8}) verwendet
werden:

\begin{verbatim}
[flake8]
ignore = E126,E127,E128
max-line-length = 99
\end{verbatim}

\subsubsection{Pytest}

Zum von uns verwendeten Testing-Framework
Pytest\footnote{\url{http://pytest.org/}} gibt es ein PEP8 Plugin. Wird dieses
aktiviert, werden Style Guide Violations als Fehlerhafte Tests gewertet.
Folgende Konfiguration wird dafür in \texttt{pytest.ini} verwendet:

\begin{verbatim}
[pytest]
addopts = --pep8
pep8ignore =
    *.py E126 E127 E128
    setup.py ALL
    settings.py ALL
    urls.py ALL
    */migrations/* ALL
    */tests/* ALL
pep8maxlinelength = 99
\end{verbatim}

\end{document}
